\documentclass[a4paper,10pt,french]{report}
\usepackage[french]{babel}
\usepackage[pdftex=true]{hyperref}
\usepackage[utf8]{inputenc}
\usepackage[T1]{fontenc}
\usepackage{epstopdf}
\usepackage{graphicx,graphics}
\usepackage{amsmath,amsfonts,amssymb}
\usepackage{xcolor}
\usepackage[top=4cm, bottom=4cm, left=2.5cm, right=2.5cm]{geometry}

\makeatletter
\renewcommand{\thesection}{\@arabic\c@section}
\makeatother

\begin{document}
\author{Antoine Wacheux et Paul Marillonnet}
\title{LO21 : Rapport de projet}\maketitle

\tableofcontents



\section*{Indroduction}\label{sec:Introduction}\addcontentsline{toc}{section}{Introduction}



\section{Description de l'architecture}\label{sec:I}



	\subsection{UTManager}\label{subsec:UTManager}
	
	La classe UTManager repose sur le design pattern Singleton, comme son nom l'indique, son rôle n'est pas seulement la gestion des UV. \\
	Ce singleton met à disposition de l'utilisateur fonctionnalités nécessaires à la gestion :
	\begin{itemize}
	\item Des UV
	\item Des Branches
	\item Des Profils
	\item Du chargement des fichiers de données XML	
	\end{itemize}
	Ainsi, se placer à un niveau de gestion au dessus de ce qu'un manager d'UV pourrait offrir en terme de fonctionnalités permet de ne pas avoir multiplier les classes de manager d'entités (UVManager, BrancheManager, ProfilManager ...)
	
		\subsubsection{}
			\paragraph{}
			\paragraph{}
			\paragraph{}
		\subsubsection{}
			\paragraph{}
			\paragraph{}
			\paragraph{}
		\subsubsection{}
			\paragraph{}
			\paragraph{}
			\paragraph{}
				
	\subsection{UTStream}\label{subsec:UTStream}
	
		Il s'agit de la classe abstraite qui présente l'interface de chargement et de sauvegarde des données.
		En pratique, la classe concrête réalisant ces deux tâches est UVStreamXML. Le modèle de données XML associé est le suivant :
		\begin{verbatim}
<SectionUV>
	<uv>
		<code></code>
		<nom></nom>
		<credit type=""></credit>
		<branche></branche>
	</uv>
</SectionUV>
<SectionBranche>
	<branche>
		<sigle></sigle>
		<nom></nom>
		<PCB></PCB>
		<PSF></PSF>
	</branche>
</SectionBranche>
<SectionProfil>
	<profil>
		<nom></nom>		
		<predicat type=" "> 
			<donnee></donnee>
			<donnee></donnee>
		</predicat>
	</profil>
</SectionProfil>
<Etudiant>
	<prenom></prenom>
	<nom></nom>
</Etudiant>
		\end{verbatim}
		
	L'attribut \emph{type} de la balise \emph{credit} d'une UV permet de stocker la catégorie d'une UV.
	Les profils communs de branches et les profils spécifiques de filières sont stockés et gérés sous un même type de balise, en accord avec l'implémentation d'une classe Profil commune aux PCB et aux PSF.
	
	% Peut être une explication plus détaillée des prédicats ?
	Les prédicats sont l'implémentation de chaque condition nécessaires à la validation d'un profil.
	Ils sont bien sur pris en compte dans la fonctionnalité d'élaboration de fin de parcours d'un étudiant.	
	Pour cette balise \emph{profil}, l'attribut \emph{type} de la balise \emph{emph} permet de stocker la valeur entière d'un type énuméré de Predicat, à savoir un Prédicat correspondant à :
	\begin{enumerate}
	\item une UV obligatoire
	\item une combinaison d'UV à valider parmi une liste donnée
	\item un minimum de crédits ECTS à obtenir dans une catégorie d'UV
	\item un minimum de crédits ECTS à valider indépendamment d'une catégorie donnée
	\end{enumerate}
	Aussi, les balises de type \emph{donnee} d'un prédicat recueillent les informations nécessaires à l'interprétation d'un prédicat par UTManager.\\
	Par exemple, pour un prédicat de type 3 concernant la catégorie d'UV Techniques et Méthodes, le code XML correspondant est par exemple :
	\begin{verbatim}
<predicat type="3">
		<donnee>TM</donnee>
		<donnee>24</donnee>
	\end{verbatim}
	
	
	
	
	
		\subsubsection{}
			\paragraph{}
			\paragraph{}
			\paragraph{}
		\subsubsection{}
			\paragraph{}
			\paragraph{}
			\paragraph{}
		\subsubsection{}
			\paragraph{}
			\paragraph{}
			\paragraph{}	
	\subsection{Editeur d'UV}\label{subsec:IC}
		Une fois cochée la case "Mode Edition des UVs" de la mainwindow, il est possible d'accéder à l'éditeur d'UVs, afin de créer, de modifier ou de supprimer des UVs.
		\subsubsection{}
			\paragraph{}
			\paragraph{}
			\paragraph{}
		\subsubsection{}
			\paragraph{}
			\paragraph{}
			\paragraph{}
		\subsubsection{}
			\paragraph{}
			\paragraph{}
			\paragraph{}	
\section{Capacité de l'architecture à évoluer}\label{sec:II}
	\subsection{}\label{subsec:IIA}
		\subsubsection{}
			\paragraph{}
			\paragraph{}
			\paragraph{}
		\subsubsection{}
			\paragraph{}
			\paragraph{}
			\paragraph{}
		\subsubsection{}
			\paragraph{}
			\paragraph{}
			\paragraph{}	
	\subsection{}\label{subsec:IIB}
	
	Le design pattern Strategy, employé dans les fonctionnalités de chargement et de sauvegarde des données, laisse le choix de l'utilisation d'un type de modèle de données par l'implémentation d'une classe fille correspondante.
	Ainsi, la classe UVStreamXML a été implémentée (comme classe dérivée de UVStream), mais il aurait été aussi possible implémenter un second modèle de stockage des données (une classe UVStreamJSON, par exemple).\\\\
	Par ailleurs, les méthodes virtuelles pures \emph{prepareLoading()} et \emph{prepareSaving()} permettent l'implémentation de modèles de stockages des données qui nécessitent des phases de préparation avant la lecture ou l'écriture de données (par exemple pour la connection à une base de données).
	
	%Expliquer la méthode UTManager::lierLesElements():
	La méthode UTManager::lierLesElements() a pour rôle d'éviter les incohérences lors du chargement des données.
	Elle permet, en faisant appel à la méthode Branche::link(), de vérifier que les profils dont les sigles correspondent aux attributs Qstring PCBString (sigle du PCB) et QStringlist psfString (sigle des PSF), sont correctement chargés lors de l'initialisation de la branche, c'est à dire à l'appel de la méthode UTManager::charger().
	
	
	
	
		\subsubsection{}
			\paragraph{}
			\paragraph{}
			\paragraph{}
		\subsubsection{}
			\paragraph{}
			\paragraph{}
			\paragraph{}
		\subsubsection{}
			\paragraph{}
			\paragraph{}
			\paragraph{}	
	\subsection{Autocompletion}\label{subsec:UV}
	
	C'est cette classe qui implémente la fonctionnalité d'élaboration de fin  de parcours.
	L'étudiant devant avoir la possibilité d'exprimer son avis concernant une éventuelle inscription à une UV, le type énuméré \emph{ExigenceUV}, dont la valeur entière s'étend de $-2$ à $2$, représente ce choix : 
	\begin{description}
	\item[-2] L'étudiant choisit de ne pas être inscrit à l'UV ;
	\item[-1] L'étudiant voudrait si possible ne pas être inscrit à l'UV ;	
	\item[0] L'étudiant exprime un avis neutre concernant l'UV ;
	\item[1] L'étudiant voudrait si possible s'inscrire à l'UV ;
	\item[2] L'étudiant veut absolument faire cette UV.
	\end{description}
	
	% Expliquer dans la partie II la pertinence du DP Strategy
	La mise en place d'un mode d'autocomplétion relève du design pattern strategy, ce qui permet de répondre au besoin de mise en placé eventuelle d'une nouvelle méthode d'autcomplétion.
	
	
		\subsubsection{}
			\paragraph{}
			\paragraph{}
			\paragraph{}
		\subsubsection{}
			\paragraph{}
			\paragraph{}
			\paragraph{}
		\subsubsection{}
			\paragraph{}
			\paragraph{}
			\paragraph{}	
\section{}\label{sec:III}
	\subsection{}\label{subsec:IIIA}
		\subsubsection{}
			\paragraph{}
			\paragraph{}
			\paragraph{}
		\subsubsection{}
			\paragraph{}
			\paragraph{}
			\paragraph{}
		\subsubsection{}
			\paragraph{}
			\paragraph{}
			\paragraph{}	
	\subsection{}\label{subsec:IIIB}
		\subsubsection{}
			\paragraph{}
			\paragraph{}
			\paragraph{}
		\subsubsection{}
			\paragraph{}
			\paragraph{}
			\paragraph{}
		\subsubsection{}
			\paragraph{}
			\paragraph{}
			\paragraph{}	
	\subsection{}\label{subsec:IIIC}
		\subsubsection{}
			\paragraph{}
			\paragraph{}
			\paragraph{}
		\subsubsection{}
			\paragraph{}
			\paragraph{}
			\paragraph{}
		\subsubsection{}
			\paragraph{}
			\paragraph{}
			\paragraph{}
			
\section*{Conclusion}\label{sec:Conclusion}\addcontentsline{toc}{section}{Conclusion}





%\section*{Bibliographie}\label{sec:Bibliographie}\addcontentsline{toc}{section}{Bibliographie}


%\bibliographystyle{plain}
%\bibliography{biblio}

\appendix
\section*{Annexes}\label{sec:Annexes}\addcontentsline{toc}{section}{Annexes}
\end{document}
